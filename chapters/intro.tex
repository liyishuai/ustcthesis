\chapter{背景简介}
\section{数据中心网络}
现代数据中心在共享的设备资源基础上,为多个用户提供不同的服务。
为保障数据安全以及用户的独立性,数据中心向每个用户展现的是虚拟化的网络环境。
这就需要数据中心的管理者部署灵活的网络功能来实现这一要求。

由于硬件网络设备普遍灵活性受限,几乎所有的云服务商都用软件来实现其网络功能,
例如:微软、亚马逊、VMWare等。但软件实现网络功能的性能天然存在两个问题:

其一,吞吐率受限。现有的软件定义网络功能通常需要使用两个以上核才能达到10Gbps带宽,
而最新的网络设备带宽已达到40\textasciitilde 100Gbps。
尽管增加单服务器的核数能够提升部分性能,但与此同时成本也会大幅提升。
一方面是设备开销提高,另一方面则是能耗显著增加。

其二,延迟高且不稳定。现有软件定义网络的处理延迟从几十微秒到毫秒量级不等。
无法满足证券交易等对低延迟要求严苛的应用需求。
%\section{模板简介}
%测试脚注\footnote{分别编号}。
%
%\subsection{模板介绍1}
%
%\subsubsection{模板测试}
%
%\subsection{模板介绍2}
%
%\section{系统要求}
%
%\section{问题反馈}
%测试脚注\footnote{脚注2}
