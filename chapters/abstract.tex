\begin{abstract}
  灵活的软件定义网络功能对于多客户的数据中心至关重要。
  现有的基于通用服务器的软件处理网络包面临高延迟、低吞吐率等问题,
  而场效可编程门阵列具有高性能、高并行、低能耗等特点,
  更符合软件定义网络的计算需求。
  本文基于微软亚洲研究院开发的ClickNP编程工具分析、设计并测试远程直接数据存取协议,
  以验证其代替快捷外设互联标准接口进行主机与场效可编程门阵列间部分数据通信的可行性。

  \keywords{软件定义网络\zhspace{} 场效可编程门阵列\zhspace{} 远程直接数据存取}
\end{abstract}

\begin{enabstract}
  Highly flexible software network functions are critical components to multi-tenant data centers.
  However, commodity servers are faced with high latency and limited capacity when processing software packets.
  In comarison, field-programmable gate arrays show advantage in performance, parallelizability and power efficiency,
  which fits the need of software-defined networking.
  This article is to analyze, implement and evaluate the RDMA protocol with the ClickNP framework
  developed by Microsoft Research Asia,
  to verify the feasibility of RDMA to replace the PCIe channel in some host-FPGA data communications.
\enkeywords{software-defined networking, FPGA, RDMA}
\end{enabstract}
