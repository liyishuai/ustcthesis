\chapter{网络应用}
为检验FPGA加速网络协议的效果,我们基于ClickNP平台开发了五个常见的网络应用。

\section{包生成器(PktGen)及抓包器(PktCap)}
包生成器可根据不同的配置生成不同模型、不同大小的网络包,通过流量控制器,以不同的速率和猝发程度发出。

抓包器将收到的包记录为日志存储在主机上。由于记录单个日志元件无法充分利用PCIe通信带宽,
抓包器通过在FPGA内建的接收端扩放元件将收到的包根据头部散列值分散到多个日志元件中。
鉴于PCIe通道带宽低于网络的40Gbps,我们在记录包的时候只提取我们最感兴趣的16字节字段,
与4字节的时间戳一起经PCIe通道发送至主机上的日志元件。

抓包器的设计体现了CPU与FPGA协同工作的重要性。相比于FPGA而言,CPU有更大的存储空间,
且更容易访问固态硬盘等其他存储器,因而将日志元件部署在CPU上有利于提升性能。

\section{Openflow防火墙(OFW)}
我们的Openflow防火墙支持对网络流量进行精确匹配及模糊匹配两种工作方式。
精确匹配表通过布谷鸟哈希算法实现,支持128K条包头记录。
模糊匹配基于三态结合存储器(ternary content-addressable memory, TCAM)实现,
然而实现三态结合存储器的完整功能在FPGA上开销过大,
一个包含512条记录的TCAM就会占用芯片上超过50\%的资源。
因此我们在实现中取而代之采用散列三态结合存储器(hashed ternary content-addressable memory, HASH-TCAM)实现。
HASH-TCAM将表空间分解为一系列更小的散列表,每个散列表对应一个掩码位,所有输入的包在散列之前都先经过一次按位与操作。
散列表中每条记录都被赋予优先级,通过仲裁逻辑选出优先级最高的记录。
散列三态结合存储器通过在容量方面的让步,节省了大量空间开销。
在我们的实现中,HASH-TCAM采用16个不同的掩码位,支持16K条流记录,与博通Trident II大致相同。

\section{互联网安全协议网关(IPSecGW)}
软件网络功能面临的一个困难是在IPSec等计算密集型系统中CPU成为性能瓶颈。
我们搭建了一条互联网安全协议数据通路,可以对IPSec包进行AES-256-CTR加密以及SHA-1认证。

因为一个AES\_CTR元件能达到的最大带宽为27.8Gbps,而链路带宽为40Gbps,
于是我们让两个AES\_CTR元件并行地工作,以充分利用带宽。

对于SHA-1认证而言,数据被分为64字节的小块。虽然对于单个数据块的计算可以流水化,
但同一个互联网协议包的相邻数据块之间存在数据依赖,即后一个块的计算需要等待前一个块计算完成才可以开始。
倘若将这些数据块串行处理,其吞吐率低达1.07Gbps。
