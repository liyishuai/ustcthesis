\chapter{网络应用}
为检验FPGA加速网络协议的效果,我们基于ClickNP平台开发了五个常见的网络应用。

\section{包生成器(PktGen)及抓包器(PktCap)}
包生成器可根据不同的配置生成不同模型、不同大小的网络包,通过流量控制器,以不同的速率和猝发程度发出。

抓包器将收到的包记录为日志存储在主机上。由于记录单个日志元件无法充分利用PCIe通信带宽,
抓包器通过在FPGA内建的接收端扩放元件将收到的包根据头部哈希值分散到多个日志元件中。
鉴于PCIe通道带宽低于网络的40Gbps,我们在记录包的时候只提取我们最感兴趣的16字节字段,
与4字节的时间戳一起经PCIe通道发送至主机上的日志元件。

抓包器的设计体现了CPU与FPGA协同工作的重要性。相比于FPGA而言,CPU有更大的存储空间,
且更容易访问固态硬盘等其他存储器,因而将日志元件部署在CPU上有利于提升性能。


